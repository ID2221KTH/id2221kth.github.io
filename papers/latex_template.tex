\documentclass[10pt]{proc}

\begin{document}

\large{\textbf{Reading Assignment X}}\\

\large{\textbf{Authors: Your names}}\\

\section{Motivation}
Describe the motivation of the paper. \textbf{WHY} is the addressed problem interesting and important to be solved?

\section{Contributions}
Explain the main contributions of the paper. \textbf{WHAT} are the solved problems?
The authors of this paper solve this and that problem. Their main contributions include this and that. blabla ...
You may optionally use a list of contributions. For example:
\begin{itemize}
 \item First contribution
 \item Second contribution
 \item Third contribution
\end{itemize}

\section{Solution}
Very briefly explain \textbf{HOW} the authors solve the mentioned problems.

\section{Strong Points}
List at least three strong points of the paper.
Is the solution good? How good is the evaluation of the work?

\begin{itemize}
 \item S1. blablabla
 \item S2. blablabla
 \item S3. blablabla
\end{itemize}

\section{Weak Points}
List at least three weak points of the paper.
What are the disadvantages and shortcomings of the solution given by the authors? Did the authors overlook anything during the evaluation of their solution?

\begin{itemize}
 \item W1. blablabla
 \item W2. blablabla
 \item W3. blablabla
\end{itemize}

\newpage
\section{Questions}
Depending on the paper assigned, you may get further instructions, if there is something specifically interesting in a reading assignment. For example, you may be asked about comparison, differences, or similarities between the solution given in the assigned paper and other solutions that are presented in the lectures. The answers to these questions must be given in a separate page. For example:
\begin{itemize}
 \item Q1. Answer to the first question goes here.
 \item Q2. Answer to the second question goes here.
 \item Q3. Answer to the third question goes here.
\end{itemize}


\end{document}
